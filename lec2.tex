%harihiom
\documentclass{book}

\usepackage[english]{babel}
%\usepackage[utf8]{inputenc}
%\usepackage{amsmath}
\usepackage{graphicx}
\usepackage{comment}
\usepackage{float}
%\usepackage{wrapfig}
%\usepackage{titlepic}
%\usepackage[colorinlistoftodos]{todonotes}
\usepackage{graphicx,epstopdf}
\usepackage{amsmath,amssymb,amsfonts,subfigure}
\usepackage{comment}
\usepackage{algorithm}
\usepackage{algpseudocode}
\usepackage{pifont}
\usepackage[normalem]{ulem}
\usepackage[english]{babel}
\usepackage[utf8x]{inputenc}
\usepackage{graphicx}
\usepackage{calc}
\usepackage{graphicx}
\usepackage{subfigure}
\usepackage{gensymb}
\usepackage{url}
\usepackage[utf8x]{inputenc}
\usepackage{amsmath}
\usepackage{graphicx}
\graphicspath{{images/}}
\usepackage{parskip}
\usepackage{fancyhdr}
\usepackage{vmargin}
\usepackage{algorithm}
\linespread{1}
\usepackage{color}
\usepackage{cite}
\usepackage{amsmath,amssymb}
\newtheorem{claim1}{Claim}
\usepackage{algpseudocode}% http://ctan.org/pkg/algorithmicx
\usepackage[compatibility=false]{caption}% http://ctan.org/pkg/caption
\setmarginsrb{3 cm}{2.5 cm}{3 cm}{2.5 cm}{1 cm}{1.5 cm}{1 cm}{1.5 cm}

%\newtheorem{theorem}{Theorem}
%\newtheorem{lemma}{Lemma}
\setcounter{chapter}{1}
\title{Probability Ideas in Computing}

\begin{document}
%\maketitle
\chapter{The Coin Tossing Game}
\textbf{The Game}: In a contest, a participant tosses a coin. If the coin shows up tail, an amount of $100$ \$ is placed on the table in front of him, and he continues tossing. If the coin shows up a head, an amount of $100$ \$ is kept on the same table and then all the money kept on the table is given to the participant concluding the game. It can be seen that the participant wins a cash amount of $100 \times number\ of\ coins\ tossed$. \\

\textbf{Question:} What is the expected value of cash price won by the participant?\\

\textbf{Answer}: 
Let $X$ be a random variable which counts the number of times the coin is tossed.\\

We know that $X$ can take the values 1, 2, 3, 4.......\\
$E[X]\ =\ 1 \times pr(X=1) + 2 \times pr(X=2) + 3 \times pr(X=3)+ .... $\\

= $\sum_{i=1}^{\infty} i \times pr(X=i) $\\

Since, $ pr(X=i) $ = $(1/2)^i$\\

Hence, $E[X]\ = \sum_{i=1}^{\infty} i/2^{n} $\\

= $1/2+(2/2^2)+(3/2^3)+(4/2^4)+.....\ =\ \alpha$, say\\

Now, $\alpha = 1/2+(2/2^2)+(3/2^3)+(4/2^4)+.....$ \hspace{5mm} ........(1)\\

$\alpha/2 = (1/2^2)+(2/2^3)+(3/2^4)+.....$\hspace{5mm} ........(2)\\

Subtracting equation (2) from equation (1) \\

$\alpha/2= (1/2)+(1/2^2)+(1/2^3)+.....$\\

It can be seen that the above expression is a geometric progression $a,ar,ar^2,ar^3......$, with $a=1$ and $r=1/2$.\\

We know that the sum of an infinite Geometric Progression = $\frac{a}{1-r}$ = $\frac{1/2}{1-1/2}$ (in this case) = $1$. \\

Hence, $\alpha/2\ = 1$\\

$\alpha=2$\\

$E[X]=2$\\  	\hspace{5mm} ........(3)

So, the expected amount of money won by the participant = $2 \times 100$ \$ = $200$ \$.\\

Now, we know that the expected amount of money won by the participant is $200$ \$. But, the standard deviation of the amount of money won (or the number of coin tosses done) can be very large\footnote{When the game is actually played, the participant can win $100$ \$ in one case, yet $500$ \$ in other case, yet $10000$ \$ in other case.}. Hence, it is important to look at the standard deviation of the random variable $X$, in addition to its expected value.\\

The formula for Standard Deviation, $\sigma(X)$, is given as :\\ 

$\sigma(X)$= $ \sqrt{(E[X- \mu])^2} $, where $\mu = E[X]$\\


= $ \sqrt{E[X^2 - 2 X \mu + \mu ^ 2]} $\\

= $ \sqrt{E[X^2 ]- 2 E[X] \mu + E[\mu ^ 2]} $\\

= $ \sqrt{E[X^2 ]- 2 \mu^2 + \mu ^ 2} $\\

= $ \sqrt{E[X^2 ]- \mu ^ 2} $\\

= $ \sqrt{E[X^2 ]- (E[X]) ^ 2} $\\

Hence, $\sigma(X)$=  $ \sqrt{E[X^2 ]- (E[X]) ^ 2} $	(4)\\

$X^2$ is also a random variable, which takes the values $1^2, 2^2, 3^2, 4^2..........$\\

$Pr(X^2=1^2) = Pr(X=1) = 1/2$\\

$Pr(X^2=2^2) = Pr(X=2) = 1/2^2$\\

... and so on \\

According the the expectation formula, $E[X^2] = 1^2 \times \frac{1}{2} + 2^2 \times \frac{1}{2^2} + 3^2 \times \frac{1}{2^3} + ............  $ = $\alpha$ say\\

$\alpha\ =\ \frac{1^2}{2^1} + \frac{2^2}{2^2} + \frac{3^2}{2^3} + \frac{4^2}{2^4} + .... $   \hspace{5mm} ........(5) \\	


$\alpha/2\ =\ \frac{1^2}{2^2} + \frac{2^2}{2^3} + \frac{3^2}{2^4} + \frac{4^2}{2^5} + .... $   \hspace{5mm} ........(6)  \\

Subtracting equation (6) from equation (5)\\

$\alpha/2\ =\ \frac{1^2}{2^1} + \frac{2^2-1^2}{2^2} + \frac{3^2-2^2}{2^3} + \frac{4^2-3^2}{2^4} + .... $    \\


or, $\alpha/2\ =\ \frac{1^2}{2^1} + \frac{(2+1)(2-1)}{2^2} + \frac{(3+2)(3-2)}{2^3} + \frac{(4+3)(4-3)}{2^4} + .... $    \\


or, $\alpha/2\ =\ \frac{1}{2^1} + \frac{3}{2^2} + \frac{5}{2^3} + \frac{7}{2^4} + .... $     \hspace{5mm} ........(7)\\	

Dividing equation (7) by equation (2)\\

 or, $\alpha/4\ =\ \frac{1}{2^2} + \frac{3}{2^3} + \frac{5}{2^4} + \frac{7}{2^5} + .... $  \hspace{5mm} ........(8)\\
 
 Subtracting equation (8) from equation (7)\\
 
$\alpha/4\ =\ \frac{1}{2} + \frac{2}{2^2} + \frac{2}{2^3} + \frac{2}{2^4} + .... $\\

or, $\alpha/4\ =\ 1 + \frac{1}{2^2} + \frac{1}{2^3} + \frac{1}{2^4} + .. $\\ 

or, $\alpha/4\ =\frac{3}{2}$, (applying the formula for sum of Geometric Progression)\\

$\alpha = 6$, \\

$E[X^2]=6$	\hspace{5mm} ........(9)\\

Substituting equation (3) and equation (9) in equation (4), \\

$\sigma(X)$=  $ \sqrt{6- 2 ^ 2} $\\

= $\sqrt{2}$\\

	
\textbf{Brain Teasers}
\begin{enumerate}
\item State whether series is convergent $\sum_{n=1}^{\infty} n/2^{n} $
\item State whether the series is convergent $\sum_{n=1}^{\infty} 1/n$ 
\end{enumerate}



\bibliographystyle{plain}
\bibliography{book}

\end{document}
